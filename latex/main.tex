\documentclass{article} % For LaTeX2e
\usepackage{iclr2019_conference,times}
%%%%% NEW MATH DEFINITIONS %%%%%

\usepackage{amsmath,amsfonts,bm}

% Mark sections of captions for referring to divisions of figures
\newcommand{\figleft}{{\em (Left)}}
\newcommand{\figcenter}{{\em (Center)}}
\newcommand{\figright}{{\em (Right)}}
\newcommand{\figtop}{{\em (Top)}}
\newcommand{\figbottom}{{\em (Bottom)}}
\newcommand{\captiona}{{\em (a)}}
\newcommand{\captionb}{{\em (b)}}
\newcommand{\captionc}{{\em (c)}}
\newcommand{\captiond}{{\em (d)}}

% Highlight a newly defined term
\newcommand{\newterm}[1]{{\bf #1}}


% Figure reference, lower-case.
\def\figref#1{figure~\ref{#1}}
% Figure reference, capital. For start of sentence
\def\Figref#1{Figure~\ref{#1}}
\def\twofigref#1#2{figures \ref{#1} and \ref{#2}}
\def\quadfigref#1#2#3#4{figures \ref{#1}, \ref{#2}, \ref{#3} and \ref{#4}}
% Section reference, lower-case.
\def\secref#1{section~\ref{#1}}
% Section reference, capital.
\def\Secref#1{Section~\ref{#1}}
% Reference to two sections.
\def\twosecrefs#1#2{sections \ref{#1} and \ref{#2}}
% Reference to three sections.
\def\secrefs#1#2#3{sections \ref{#1}, \ref{#2} and \ref{#3}}
% Reference to an equation, lower-case.
\def\eqref#1{equation~\ref{#1}}
% Reference to an equation, upper case
\def\Eqref#1{Equation~\ref{#1}}
% A raw reference to an equation---avoid using if possible
\def\plaineqref#1{\ref{#1}}
% Reference to a chapter, lower-case.
\def\chapref#1{chapter~\ref{#1}}
% Reference to an equation, upper case.
\def\Chapref#1{Chapter~\ref{#1}}
% Reference to a range of chapters
\def\rangechapref#1#2{chapters\ref{#1}--\ref{#2}}
% Reference to an algorithm, lower-case.
\def\algref#1{algorithm~\ref{#1}}
% Reference to an algorithm, upper case.
\def\Algref#1{Algorithm~\ref{#1}}
\def\twoalgref#1#2{algorithms \ref{#1} and \ref{#2}}
\def\Twoalgref#1#2{Algorithms \ref{#1} and \ref{#2}}
% Reference to a part, lower case
\def\partref#1{part~\ref{#1}}
% Reference to a part, upper case
\def\Partref#1{Part~\ref{#1}}
\def\twopartref#1#2{parts \ref{#1} and \ref{#2}}

\def\ceil#1{\lceil #1 \rceil}
\def\floor#1{\lfloor #1 \rfloor}
\def\1{\bm{1}}
\newcommand{\train}{\mathcal{D}}
\newcommand{\valid}{\mathcal{D_{\mathrm{valid}}}}
\newcommand{\test}{\mathcal{D_{\mathrm{test}}}}

\def\eps{{\epsilon}}


% Random variables
\def\reta{{\textnormal{$\eta$}}}
\def\ra{{\textnormal{a}}}
\def\rb{{\textnormal{b}}}
\def\rc{{\textnormal{c}}}
\def\rd{{\textnormal{d}}}
\def\re{{\textnormal{e}}}
\def\rf{{\textnormal{f}}}
\def\rg{{\textnormal{g}}}
\def\rh{{\textnormal{h}}}
\def\ri{{\textnormal{i}}}
\def\rj{{\textnormal{j}}}
\def\rk{{\textnormal{k}}}
\def\rl{{\textnormal{l}}}
% rm is already a command, just don't name any random variables m
\def\rn{{\textnormal{n}}}
\def\ro{{\textnormal{o}}}
\def\rp{{\textnormal{p}}}
\def\rq{{\textnormal{q}}}
\def\rr{{\textnormal{r}}}
\def\rs{{\textnormal{s}}}
\def\rt{{\textnormal{t}}}
\def\ru{{\textnormal{u}}}
\def\rv{{\textnormal{v}}}
\def\rw{{\textnormal{w}}}
\def\rx{{\textnormal{x}}}
\def\ry{{\textnormal{y}}}
\def\rz{{\textnormal{z}}}

% Random vectors
\def\rvepsilon{{\mathbf{\epsilon}}}
\def\rvtheta{{\mathbf{\theta}}}
\def\rva{{\mathbf{a}}}
\def\rvb{{\mathbf{b}}}
\def\rvc{{\mathbf{c}}}
\def\rvd{{\mathbf{d}}}
\def\rve{{\mathbf{e}}}
\def\rvf{{\mathbf{f}}}
\def\rvg{{\mathbf{g}}}
\def\rvh{{\mathbf{h}}}
\def\rvu{{\mathbf{i}}}
\def\rvj{{\mathbf{j}}}
\def\rvk{{\mathbf{k}}}
\def\rvl{{\mathbf{l}}}
\def\rvm{{\mathbf{m}}}
\def\rvn{{\mathbf{n}}}
\def\rvo{{\mathbf{o}}}
\def\rvp{{\mathbf{p}}}
\def\rvq{{\mathbf{q}}}
\def\rvr{{\mathbf{r}}}
\def\rvs{{\mathbf{s}}}
\def\rvt{{\mathbf{t}}}
\def\rvu{{\mathbf{u}}}
\def\rvv{{\mathbf{v}}}
\def\rvw{{\mathbf{w}}}
\def\rvx{{\mathbf{x}}}
\def\rvy{{\mathbf{y}}}
\def\rvz{{\mathbf{z}}}

% Elements of random vectors
\def\erva{{\textnormal{a}}}
\def\ervb{{\textnormal{b}}}
\def\ervc{{\textnormal{c}}}
\def\ervd{{\textnormal{d}}}
\def\erve{{\textnormal{e}}}
\def\ervf{{\textnormal{f}}}
\def\ervg{{\textnormal{g}}}
\def\ervh{{\textnormal{h}}}
\def\ervi{{\textnormal{i}}}
\def\ervj{{\textnormal{j}}}
\def\ervk{{\textnormal{k}}}
\def\ervl{{\textnormal{l}}}
\def\ervm{{\textnormal{m}}}
\def\ervn{{\textnormal{n}}}
\def\ervo{{\textnormal{o}}}
\def\ervp{{\textnormal{p}}}
\def\ervq{{\textnormal{q}}}
\def\ervr{{\textnormal{r}}}
\def\ervs{{\textnormal{s}}}
\def\ervt{{\textnormal{t}}}
\def\ervu{{\textnormal{u}}}
\def\ervv{{\textnormal{v}}}
\def\ervw{{\textnormal{w}}}
\def\ervx{{\textnormal{x}}}
\def\ervy{{\textnormal{y}}}
\def\ervz{{\textnormal{z}}}

% Random matrices
\def\rmA{{\mathbf{A}}}
\def\rmB{{\mathbf{B}}}
\def\rmC{{\mathbf{C}}}
\def\rmD{{\mathbf{D}}}
\def\rmE{{\mathbf{E}}}
\def\rmF{{\mathbf{F}}}
\def\rmG{{\mathbf{G}}}
\def\rmH{{\mathbf{H}}}
\def\rmI{{\mathbf{I}}}
\def\rmJ{{\mathbf{J}}}
\def\rmK{{\mathbf{K}}}
\def\rmL{{\mathbf{L}}}
\def\rmM{{\mathbf{M}}}
\def\rmN{{\mathbf{N}}}
\def\rmO{{\mathbf{O}}}
\def\rmP{{\mathbf{P}}}
\def\rmQ{{\mathbf{Q}}}
\def\rmR{{\mathbf{R}}}
\def\rmS{{\mathbf{S}}}
\def\rmT{{\mathbf{T}}}
\def\rmU{{\mathbf{U}}}
\def\rmV{{\mathbf{V}}}
\def\rmW{{\mathbf{W}}}
\def\rmX{{\mathbf{X}}}
\def\rmY{{\mathbf{Y}}}
\def\rmZ{{\mathbf{Z}}}

% Elements of random matrices
\def\ermA{{\textnormal{A}}}
\def\ermB{{\textnormal{B}}}
\def\ermC{{\textnormal{C}}}
\def\ermD{{\textnormal{D}}}
\def\ermE{{\textnormal{E}}}
\def\ermF{{\textnormal{F}}}
\def\ermG{{\textnormal{G}}}
\def\ermH{{\textnormal{H}}}
\def\ermI{{\textnormal{I}}}
\def\ermJ{{\textnormal{J}}}
\def\ermK{{\textnormal{K}}}
\def\ermL{{\textnormal{L}}}
\def\ermM{{\textnormal{M}}}
\def\ermN{{\textnormal{N}}}
\def\ermO{{\textnormal{O}}}
\def\ermP{{\textnormal{P}}}
\def\ermQ{{\textnormal{Q}}}
\def\ermR{{\textnormal{R}}}
\def\ermS{{\textnormal{S}}}
\def\ermT{{\textnormal{T}}}
\def\ermU{{\textnormal{U}}}
\def\ermV{{\textnormal{V}}}
\def\ermW{{\textnormal{W}}}
\def\ermX{{\textnormal{X}}}
\def\ermY{{\textnormal{Y}}}
\def\ermZ{{\textnormal{Z}}}

% Vectors
\def\vzero{{\bm{0}}}
\def\vone{{\bm{1}}}
\def\vmu{{\bm{\mu}}}
\def\vtheta{{\bm{\theta}}}
\def\va{{\bm{a}}}
\def\vb{{\bm{b}}}
\def\vc{{\bm{c}}}
\def\vd{{\bm{d}}}
\def\ve{{\bm{e}}}
\def\vf{{\bm{f}}}
\def\vg{{\bm{g}}}
\def\vh{{\bm{h}}}
\def\vi{{\bm{i}}}
\def\vj{{\bm{j}}}
\def\vk{{\bm{k}}}
\def\vl{{\bm{l}}}
\def\vm{{\bm{m}}}
\def\vn{{\bm{n}}}
\def\vo{{\bm{o}}}
\def\vp{{\bm{p}}}
\def\vq{{\bm{q}}}
\def\vr{{\bm{r}}}
\def\vs{{\bm{s}}}
\def\vt{{\bm{t}}}
\def\vu{{\bm{u}}}
\def\vv{{\bm{v}}}
\def\vw{{\bm{w}}}
\def\vx{{\bm{x}}}
\def\vy{{\bm{y}}}
\def\vz{{\bm{z}}}

% Elements of vectors
\def\evalpha{{\alpha}}
\def\evbeta{{\beta}}
\def\evepsilon{{\epsilon}}
\def\evlambda{{\lambda}}
\def\evomega{{\omega}}
\def\evmu{{\mu}}
\def\evpsi{{\psi}}
\def\evsigma{{\sigma}}
\def\evtheta{{\theta}}
\def\eva{{a}}
\def\evb{{b}}
\def\evc{{c}}
\def\evd{{d}}
\def\eve{{e}}
\def\evf{{f}}
\def\evg{{g}}
\def\evh{{h}}
\def\evi{{i}}
\def\evj{{j}}
\def\evk{{k}}
\def\evl{{l}}
\def\evm{{m}}
\def\evn{{n}}
\def\evo{{o}}
\def\evp{{p}}
\def\evq{{q}}
\def\evr{{r}}
\def\evs{{s}}
\def\evt{{t}}
\def\evu{{u}}
\def\evv{{v}}
\def\evw{{w}}
\def\evx{{x}}
\def\evy{{y}}
\def\evz{{z}}

% Matrix
\def\mA{{\bm{A}}}
\def\mB{{\bm{B}}}
\def\mC{{\bm{C}}}
\def\mD{{\bm{D}}}
\def\mE{{\bm{E}}}
\def\mF{{\bm{F}}}
\def\mG{{\bm{G}}}
\def\mH{{\bm{H}}}
\def\mI{{\bm{I}}}
\def\mJ{{\bm{J}}}
\def\mK{{\bm{K}}}
\def\mL{{\bm{L}}}
\def\mM{{\bm{M}}}
\def\mN{{\bm{N}}}
\def\mO{{\bm{O}}}
\def\mP{{\bm{P}}}
\def\mQ{{\bm{Q}}}
\def\mR{{\bm{R}}}
\def\mS{{\bm{S}}}
\def\mT{{\bm{T}}}
\def\mU{{\bm{U}}}
\def\mV{{\bm{V}}}
\def\mW{{\bm{W}}}
\def\mX{{\bm{X}}}
\def\mY{{\bm{Y}}}
\def\mZ{{\bm{Z}}}
\def\mBeta{{\bm{\beta}}}
\def\mPhi{{\bm{\Phi}}}
\def\mLambda{{\bm{\Lambda}}}
\def\mSigma{{\bm{\Sigma}}}

% Tensor
\DeclareMathAlphabet{\mathsfit}{\encodingdefault}{\sfdefault}{m}{sl}
\SetMathAlphabet{\mathsfit}{bold}{\encodingdefault}{\sfdefault}{bx}{n}
\newcommand{\tens}[1]{\bm{\mathsfit{#1}}}
\def\tA{{\tens{A}}}
\def\tB{{\tens{B}}}
\def\tC{{\tens{C}}}
\def\tD{{\tens{D}}}
\def\tE{{\tens{E}}}
\def\tF{{\tens{F}}}
\def\tG{{\tens{G}}}
\def\tH{{\tens{H}}}
\def\tI{{\tens{I}}}
\def\tJ{{\tens{J}}}
\def\tK{{\tens{K}}}
\def\tL{{\tens{L}}}
\def\tM{{\tens{M}}}
\def\tN{{\tens{N}}}
\def\tO{{\tens{O}}}
\def\tP{{\tens{P}}}
\def\tQ{{\tens{Q}}}
\def\tR{{\tens{R}}}
\def\tS{{\tens{S}}}
\def\tT{{\tens{T}}}
\def\tU{{\tens{U}}}
\def\tV{{\tens{V}}}
\def\tW{{\tens{W}}}
\def\tX{{\tens{X}}}
\def\tY{{\tens{Y}}}
\def\tZ{{\tens{Z}}}


% Graph
\def\gA{{\mathcal{A}}}
\def\gB{{\mathcal{B}}}
\def\gC{{\mathcal{C}}}
\def\gD{{\mathcal{D}}}
\def\gE{{\mathcal{E}}}
\def\gF{{\mathcal{F}}}
\def\gG{{\mathcal{G}}}
\def\gH{{\mathcal{H}}}
\def\gI{{\mathcal{I}}}
\def\gJ{{\mathcal{J}}}
\def\gK{{\mathcal{K}}}
\def\gL{{\mathcal{L}}}
\def\gM{{\mathcal{M}}}
\def\gN{{\mathcal{N}}}
\def\gO{{\mathcal{O}}}
\def\gP{{\mathcal{P}}}
\def\gQ{{\mathcal{Q}}}
\def\gR{{\mathcal{R}}}
\def\gS{{\mathcal{S}}}
\def\gT{{\mathcal{T}}}
\def\gU{{\mathcal{U}}}
\def\gV{{\mathcal{V}}}
\def\gW{{\mathcal{W}}}
\def\gX{{\mathcal{X}}}
\def\gY{{\mathcal{Y}}}
\def\gZ{{\mathcal{Z}}}

% Sets
\def\sA{{\mathbb{A}}}
\def\sB{{\mathbb{B}}}
\def\sC{{\mathbb{C}}}
\def\sD{{\mathbb{D}}}
% Don't use a set called E, because this would be the same as our symbol
% for expectation.
\def\sF{{\mathbb{F}}}
\def\sG{{\mathbb{G}}}
\def\sH{{\mathbb{H}}}
\def\sI{{\mathbb{I}}}
\def\sJ{{\mathbb{J}}}
\def\sK{{\mathbb{K}}}
\def\sL{{\mathbb{L}}}
\def\sM{{\mathbb{M}}}
\def\sN{{\mathbb{N}}}
\def\sO{{\mathbb{O}}}
\def\sP{{\mathbb{P}}}
\def\sQ{{\mathbb{Q}}}
\def\sR{{\mathbb{R}}}
\def\sS{{\mathbb{S}}}
\def\sT{{\mathbb{T}}}
\def\sU{{\mathbb{U}}}
\def\sV{{\mathbb{V}}}
\def\sW{{\mathbb{W}}}
\def\sX{{\mathbb{X}}}
\def\sY{{\mathbb{Y}}}
\def\sZ{{\mathbb{Z}}}

% Entries of a matrix
\def\emLambda{{\Lambda}}
\def\emA{{A}}
\def\emB{{B}}
\def\emC{{C}}
\def\emD{{D}}
\def\emE{{E}}
\def\emF{{F}}
\def\emG{{G}}
\def\emH{{H}}
\def\emI{{I}}
\def\emJ{{J}}
\def\emK{{K}}
\def\emL{{L}}
\def\emM{{M}}
\def\emN{{N}}
\def\emO{{O}}
\def\emP{{P}}
\def\emQ{{Q}}
\def\emR{{R}}
\def\emS{{S}}
\def\emT{{T}}
\def\emU{{U}}
\def\emV{{V}}
\def\emW{{W}}
\def\emX{{X}}
\def\emY{{Y}}
\def\emZ{{Z}}
\def\emSigma{{\Sigma}}

% entries of a tensor
% Same font as tensor, without \bm wrapper
\newcommand{\etens}[1]{\mathsfit{#1}}
\def\etLambda{{\etens{\Lambda}}}
\def\etA{{\etens{A}}}
\def\etB{{\etens{B}}}
\def\etC{{\etens{C}}}
\def\etD{{\etens{D}}}
\def\etE{{\etens{E}}}
\def\etF{{\etens{F}}}
\def\etG{{\etens{G}}}
\def\etH{{\etens{H}}}
\def\etI{{\etens{I}}}
\def\etJ{{\etens{J}}}
\def\etK{{\etens{K}}}
\def\etL{{\etens{L}}}
\def\etM{{\etens{M}}}
\def\etN{{\etens{N}}}
\def\etO{{\etens{O}}}
\def\etP{{\etens{P}}}
\def\etQ{{\etens{Q}}}
\def\etR{{\etens{R}}}
\def\etS{{\etens{S}}}
\def\etT{{\etens{T}}}
\def\etU{{\etens{U}}}
\def\etV{{\etens{V}}}
\def\etW{{\etens{W}}}
\def\etX{{\etens{X}}}
\def\etY{{\etens{Y}}}
\def\etZ{{\etens{Z}}}

% The true underlying data generating distribution
\newcommand{\pdata}{p_{\rm{data}}}
% The empirical distribution defined by the training set
\newcommand{\ptrain}{\hat{p}_{\rm{data}}}
\newcommand{\Ptrain}{\hat{P}_{\rm{data}}}
% The model distribution
\newcommand{\pmodel}{p_{\rm{model}}}
\newcommand{\Pmodel}{P_{\rm{model}}}
\newcommand{\ptildemodel}{\tilde{p}_{\rm{model}}}
% Stochastic autoencoder distributions
\newcommand{\pencode}{p_{\rm{encoder}}}
\newcommand{\pdecode}{p_{\rm{decoder}}}
\newcommand{\precons}{p_{\rm{reconstruct}}}

\newcommand{\laplace}{\mathrm{Laplace}} % Laplace distribution

\newcommand{\E}{\mathbb{E}}
\newcommand{\Ls}{\mathcal{L}}
\newcommand{\R}{\mathbb{R}}
\newcommand{\emp}{\tilde{p}}
\newcommand{\lr}{\alpha}
\newcommand{\reg}{\lambda}
\newcommand{\rect}{\mathrm{rectifier}}
\newcommand{\softmax}{\mathrm{softmax}}
\newcommand{\sigmoid}{\sigma}
\newcommand{\softplus}{\zeta}
\newcommand{\KL}{D_{\mathrm{KL}}}
\newcommand{\Var}{\mathrm{Var}}
\newcommand{\standarderror}{\mathrm{SE}}
\newcommand{\Cov}{\mathrm{Cov}}
% Wolfram Mathworld says $L^2$ is for function spaces and $\ell^2$ is for vectors
% But then they seem to use $L^2$ for vectors throughout the site, and so does
% wikipedia.
\newcommand{\normlzero}{L^0}
\newcommand{\normlone}{L^1}
\newcommand{\normltwo}{L^2}
\newcommand{\normlp}{L^p}
\newcommand{\normmax}{L^\infty}

\newcommand{\parents}{Pa} % See usage in notation.tex. Chosen to match Daphne's book.

\DeclareMathOperator*{\argmax}{arg\,max}
\DeclareMathOperator*{\argmin}{arg\,min}

\DeclareMathOperator{\sign}{sign}
\DeclareMathOperator{\Tr}{Tr}
\let\ab\allowbreak

\usepackage{hyperref}
\usepackage{url}
\usepackage{graphicx}
\usepackage{xurl}
\usepackage{mathtools}

\graphicspath{ {./img/} }

\title{Explicitly disentangling image content from translation and rotation with spatial-VAE}

\author{Yanislav Donchev Donchev, Joe Simon \& Pier Paolo Ippolito \\
\texttt{\{ydd1g16,js23g16,ppi1u16\}@ecs.soton.ac.uk}
}

\newcommand{\fix}{\marginpar{FIX}}
\newcommand{\new}{\marginpar{NEW}}

\iclrfinalcopy % Uncomment for camera-ready version, but NOT for submission.
\begin{document}
\maketitle
% \begin{abstract}
% %done pls review
% Image generative models have become increasingly popular for performing supervised and unsupervised learning various imaging domains. Real-world images come with geometric pose transformations applied to them making it challenging for models to train on them. Different approaches have been taken to explicitly learn these transformations to make them pose invariant.
% In this report, we reproduce the implementation of the NeurIPS 2019 paper, SpatialVAE which attempts to learn the rotation and translation pose transformations and critique its reproducibility by comparing results and judging its difficulty. 
% % % Tag along summary of conclusion.


% \end{abstract}

% A good reproducibility report, describes the target questions, the experimental methodology, the implementation details, provides analysis and, discusses findings and conclusions on the reproducibility of the paper. The result of the reproducibility study should NOT be a simple Pass / Fail outcome. The goal should be to identify which parts of the contribution can be reproduced, and at what cost in terms of resources (computation, time, people, development effort, etc). Other than briefly outlining the core ideas or approach of the original paper, there is no need to repeat information.

\section{Introduction} \label{introduction}
% The goal is to assess if the experiments are reproducible, and to determine if the conclusions of the paper are supported by your findings. Your results can be either positive (i.e. confirm reproducibility).

% Essentially, think of your role as an inspector verifying the validity of the experimental results and conclusions of the paper.

% A good reproducibility report, describes the target questions, the experimental methodology, the implementation details, provides analysis and, discusses findings and conclusions on the reproducibility of the paper. The result of the reproducibility study should NOT be a simple Pass / Fail outcome. The goal should be to identify which parts of the contribution can be reproduced, and at what cost in terms of resources (computation, time, people, development effort, etc). Other than briefly outlining the core ideas or approach of the original paper, there is no need to repeat information.

% Ideally you should include a copy of your report in your git repository as this will serve as a useful guide for others.

% You should make it clear to what extent you used existing code (e.g. that of the authors’ of your chosen paper) compared to your own code.

%Set and describe the problem
Common issues when training models using real images are the inconsistencies in the pose of the captured image. For example, a famous landmark could be captured by tourists from different angles, elevations and shifts in position, resulting in different looking images of the same object relative to a fixed position. When learning a low-dimensional latent representation of these images, these inconsistencies in the pose might be mixed up with the actual content of the image in the latent space.

% This changing geometric pose problem can be found when observing planetary systems from different observatories around the world, studying scattered protein structures under electron microscopes and tracking moving objects (i.e. self-driving car reading a signal).

% Describe the solution of the original paper
Spatial-VAE (SVAE) is a variational autoencoder framework that seeks to learn the rotation and translation pose variables (or any other linear transformation) separately from the continuous latent space variables in an unsupervised manner. Unlike a vanilla\footnote{The standard VAE architecture is referred to as \textit{vanilla}.} VAE (VVAE), where the decoder outputs the whole image at once, the decoder of an SVAE outputs a single-pixel given the pixel's spatial coordinates. Apart from the normal latent variables of a VVAE, the encoder of an SVAE learns a set of transformation parameters with which the input pixel coordinates are transformed, thus ``explicitly disentangling the content from linear transformations'' of the input image.

In this report, we reproduce the implementation of the NeurIPS 2019 paper - SpatialVAE, which attempts to learn the rotation and translation pose transformations and critique its reproducibility by comparing results and judging its difficulty. Although the authors provide code\footnote{Original implementation: https://github.com/tbepler/spatial-VAE.} for their paper, we reimplemented their work\footnote{Our implementation: https://github.com/COMP6248-Reproducability-Challenge/SVAE.} based on the paper alone using PyTorch and PyTorch Lightning.

%Describe the structure of our paper.
% In this report, we attempt to re-describe the implementations of the VAE as described in Section \ref{implandexp} and reconstruct the framework described from scratch in PyTorch\footnote{our codebase link} without referencing the original codebase. The reproduced results of the MNIST dataset and its pose transformed variants are then compared with the original in Section \ref{resultsandeval}. 

\section{Implementation Details} \label{impl}
% The methods described can also be implemented/re-implemented according to the description in the paper. This is a higher bar for reproducibility, but may be helpful in detecting anomalies in the code, or shedding light on aspects of the implementation that affect results.

The SVAE architecture (see Figure \ref{fig:arch}) is very similar to a VVAE with a few notable changes. First, the encoder models a few extra distributions that encode the transformation parameters, in addition to the traditional unconstrained latent variables. The sampling (or reparameterisation) is similar to a VVAE and uses the typical prior $\mathcal{N}(0, I)$ for the unconstrained latent variables $\textbf{z}$. Since it is likely to have prior knowledge of the transformations, a separate prior is defined for these. Due to the rotation of the image $\theta$ being bounded, the authors define an adjustment to the standard KL divergence (equation (3) in \cite{bepler2019spatialvae}). The biggest difference from a VVAE is in the decoder: instead of accepting only $\textbf{z}$, the decoder also takes a two-dimensional spatial pixel coordinate $\textbf{x}$, which has been linearly transformed based on the sampled latent space variables for rotation ($\theta$) and translation ($\Delta \textbf{x}$). The input to the decoder is the concatenated vector $(\textbf{z}, \textbf{x})$, where $\textbf{x} = (T(\theta, \Delta \textbf{x}) \cdot \textbf{x}^{\prime})_{1:2}$, and the transformation matrix and non-transformed spatial coordinates are defined in (\ref{eq:transf}) and (\ref{eq:coords}) respectively. This operation is done for a normalised meshgrid ($28 \times 28$ in the case of MNIST) to reconstruct the full image.

\begin{equation}
\label{eq:transf}
T(\theta, \Delta \textbf{x}) = 
\begin{bmatrix}
cos(\theta) & -sin(\theta) & \Delta x_0\\
sin(\theta) & cos(\theta) & \Delta x_1\\
0 & 0 & 1
\end{bmatrix}
\end{equation}

\begin{equation}
\label{eq:coords}
\textbf{x}^{\prime} = 
\begin{bmatrix}
x_0 & x_1 & 1
\end{bmatrix}^T
\end{equation}

where $x_0, x_1 \in [-1, 1]$.

\begin{figure}[t]
    \begin{center}
        \includegraphics[width=\textwidth]{"svae.pdf"}
    \end{center}
    \caption{Diagram representation of the spatial-VAE framework. The data flow is clearly defined to guide implementation. A notable difference from a VVAE is that the network outputs a single-pixel value $y$ instead of the full image, and the decoder takes in an extra input - the transformed spatial pixel coordinate $\textbf{x}$. The output is squished by a sigmoid.}
    \label{fig:arch}
\end{figure}

% Additionally, in order to correctly implement the function to be optimised during training, different literature papers such as \cite{bepler2019learning} and \cite{kingma2014stochastic} have been analysed.

\section{Experimental Methodology} \label{exp}

To test their hypothesis -- can the network explicitly differentiate linear transformations from content -- the authors use the MNIST dataset and two variations of it. While the standard MNIST dataset provides a good baseline, the two additional variants -- rotated MNIST and rotated \& translated MNIST -- help in observing the network's performance on pose transformed images. The rotated MNIST has randomly sampled angles from $\mathcal{N}(0, \pi^2/16)$ applied to the images along with a small random translation from $\mathcal{N}(0, 1.4^2)$. The more challenging rotated \& translated MNIST dataset maintained the same degree of rotation but added a greater degree of translation, randomly sampled from $\mathcal{N}(0, 14^2)$. 

The original paper also tests the framework on the \textit{Sloan Digital Sky Survey Galaxy Zoo}, and \textit{Single Particle Electron-Microscopy} to address the challenges mentioned in Section \ref{introduction}. This report focuses solely on experimentation on the MNIST dataset and its variants due to the size and GPU hardware requirements of the latter datasets.

Three VAE models were tested for comparison: VVAE acting as a baseline, SVAE and SVAE with $\theta$ and $\Delta \textbf{x}$ set to zero. Four different latent dimensions Z-D were tested for each of the models above: Z-D $\in \{ 2, 3, 5, 10 \}$. The encoders and decoders are standard two-layer MLPs with 500 neural units each and tanh activations. 

The SVAE employs the same parameters but it has three additional latent variables -- two that encode translation and one for rotation. It outputs a single probability to represent binary pixels. Its rotation prior for the regular MNIST is set to $\mathcal{N}(0, \pi^2/64)$, while the rotation prior is set to $\mathcal{N}(0, \pi^2/16)$ for the transformed MNIST datasets. The translation prior is set to $\mathcal{N}(0, 1.4^2)$ for the transformed datasets and set to a constant of zero for the regular MNIST. The second SVAE model with $\theta=0$ and $\Delta \textbf{x}=0$ is identical to the above model with the exception that the input coordinates $\textbf{x}^{\prime}$ are not transformed (i.e. multiplied by $I$).

All models are trained with the ADAM optimiser, learning rate of 1-e4 and a minibatch size of 100. The loss function aims to maximise the Evidence Lower Bound (ELBO). Every combination was run for 200 epochs.


\section{Results \& Evaluation} \label{resultsandeval}
A total of 36 networks were trained with total runtime of 54 hours. The VVAE models took approximately 0.5 hours each to train, while the SVAE models took significantly longer at approximately 2 hours. We trained the networks in parallel on 9 separate GPUs (Nvidia GTX 2070 and Nvidia P100) from Kaggle and Google Colab. % While they were able to handle MNIST, larger datasets would require more GPUS and/or more time to process, potentially becoming expensive.

\begin{figure}[t]
    \begin{center}
        \includegraphics[width=\textwidth]{"elbos.pdf"}
    \end{center}
    \caption{Comparison of ELBO logs of the VAE models on the MNIST flavours while varying the dimensions of the unstructured latent variables. The lines show training (solid) and testing (dashed) ELBO for the VVAE model (blue), SVAE model (green) and SVAE with $\theta=0$ and $\Delta \textbf{x}=0$ (orange).}
    \label{fig:elbo}
\end{figure}

The ELBO logs of these runs are shown in Figure \ref{fig:elbo}. The plots quite closely resemble the images shown in Figure 2 in \cite{bepler2019spatialvae}. The performance of the SVAE is well highlighted when the latent space dimension is low, and even more so in the more challenging rotated MNIST and rotated & translated MNIST datasets. It becomes apparent that the positive effect of SVAE on ELBO diminishes as Z-D grows, and VVAE achieves similar results. Overall, the differences with the original results are marginal and could be the effect of a different random initialisation. 

Similarly, the latent space manifolds of the models are reproduced in Figure \ref{fig:manifold} and seem to be in line with Figure 3 in \cite{bepler2019spatialvae}, confirming that the described framework is reproducible. The benefit of SVAE becomes apparent as it can capture the content of the heavily transformed MNIST datasets, while VVAE fails to do so.

To make the benefit of SVAE more intuitive, we converted our model that was trained on the rotated \& translated MNIST dataset to the Open Neural Network Exchange (ONNX) format and deployed it in an interactive webpage\footnote{Interactive demo: https://comp6248-reproducability-challenge.github.io/SVAE/}. Handwritten digits can be drawn and linearly transformed using sliders, while the model generates stable samples that correct for these transformations.

While the experiments and framework were well described allowing us to replicate it from scratch, the lack of specific model initializations and repeated runs on different random seed values prevented robust testing and prevented the measurement of uncertainty in the results.

Additionally, their codebase seemed unwieldy and hard to follow due to the lack of good coding practices and sparse comments, making it difficult for people to understand the codebase along with the paper. Our codebase aimed to rewrite the entire codebase solely based on the original paper and \cite{kingma2014stochastic}, providing a closely matched structure and good practices, allowing the readers to better understand the work. Improved commenting on the structure and the maths, and Python docstrings were added to further aid users.

% The  paper should clearly indicate the Github repository in the “COMP6248 Reproducability Challenge” organisation that accompanies the paper, which should contain the code used for the experiments in the report. 

\begin{figure}[t]
    \begin{center}
        \includegraphics[width=\textwidth]{"manifolds.pdf"}
    \end{center}
    \caption{The latent space manifolds of the three models with Z-D=2, trained on the MNIST variants. The plots show the outputs of the decoders for uniform samples of the prior.}
    \label{fig:manifold}
\end{figure}


\section{Conclusion} \label{conclusion}

Overall, the paper was written well allowing its results to be reproducible and the claims of the authors stood true -- the SVAE architecture successfully learns linear transformations of the input image, separate from the semantics. However this comes at a cost: the decoder has to be executed once for each pixel of the image, which significantly increases the runtime. Furthermore, if the claimed disentangling is not required in the application, a VVAE with a bigger latent space achieves a similar ELBO to that of the SVAE at a much lower computational cost. Additionally, an effort was put on to make the work more accessible and readable through the help of a rewritten codebase and a live interactive demo of the work.



% You should make it clear to what extent you used existing code (e.g. that of the authors’ of your chosen paper) compared to your own code.


\bibliography{iclr2019_conference}
\bibliographystyle{iclr2019_conference}

\end{document}
